\likechapter{ВВЕДЕНИЕ}

Современный мир невозможно представить без компьютера. Компьютерные технологии проникли во множество направлений деятельности человека. Если раньше персональные компьютеры были связаны напрямую с потребностями в работе, то сейчас они есть практически у каждого человека. Также с развитием содержания меняется и форма. Такое направление как “носимая электроника” почти стирает грань между техникой и человеком. Всё быстрее компании, которые сталкиваются с проблемами, касающихся использования протокола IPv4 во время обмена сообщениями, переходят на обновлённый IPv6, позволяющий буквально присвоить интернет адрес каждой песчинке на Земле.

Но и как в большинстве дел, связанных с получением дохода или информации, в компьютерной индустрии присутствуют субъекты, пытающиеся извлечь для себя максимальную прибыль. Часто не совсем законными способами. Прибыль не обязательно выражается в материальном, денежном виде. В нашем случае часто большую ценность может иметь информация. И растущая распространённость техники, контролируемой автоматически, может пугать своими потенциальными последствиями её захвата. Часто в качестве средства захвата выступает вредоносное ПО. Существует 
множество типов

Написать про типа анализа вредоносных файлов.

В первой главе мы опишем, какие файлы были включены в исследование.

Во второй главе мы рассмотрим один способов сбора информации о запущенных процессах в операционной системе “Windows”.  По сформированной выборке из файлов формата .doc мы будем собирать информацию с помощью утилиты Procmon. Также будет описан формат файла, содержащего различные данные о процессе.

Третья глава посвящена введению в теорию машинного обучения, которая нам понадобится для дальнейшей работы с созданием метода классификации вредоносных программ. В начале главы мы сформулируем главную проблему классификации и дадим необходимые определения. Так как мы хотим классифицировать файлы произвольной структуры, далее, мы рассмотрим принципы, позволяющие работать с произвольно сложными объектами в качестве субъектов машинного обучения. Также будут описаны несколько алгоритмов классификации, пользуясь которыми, мы в дальнейшем будем решать задачу распознавания вредоносных объектов. В качестве таких алгоритмов мы рассмотрим метод K ближайших соседей и логистическую регрессию.  В завершении главы мы опишем способы оценки качества полученных моделей классификации и сравним несколько алгоритмов между собой на примере задачи распознавания цветков ириса.

В четвёртой главе мы воспользуемся информацией, рассмотренной в ранних главах, и создадим набор из двух алгоритмов, позволяющих классифицировать объекты из собранной выборки.


