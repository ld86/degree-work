\likechapter{ВВЕДЕНИЕ}

Многолетнее противодействие вредоносных программ и антивирусных средств закономерно приводит к усложнению и совершенствованию обеих сторон этого непрекращающегося конфликта. В результате вредоносные программы становятся все более опасными, ухищренными и неуязвимыми. Расширяются механизмы и способы внедрения, область их применения.

Среди общего многообразия вредоносных программ следует выделить отдельный подкласс -  файлы документов распространенных и сложных форматов: \textbf{doc}, \textbf{pdf}, \textbf{jpg} и др., которые при открытии в собственных приложениях реализуют атаку на переполнение буфера. Они способны обходить традиционные формы антивирусной защиты и не выявляются в автоматическом режиме. Обнаружение подобных вредоносных программ возможно при проведении тщательного экспертного исследования компьютерной системы с использованием специальных программных средств. Такие исследования требуют больших временных затрат и высокой квалификации эксперта. При этом выявляемые компоненты могут не содержать собственно программного кода в бинарном либо текстовом видах. Очевидна необходимость разработки методик исследования и соответствующего программного инструментария для анализа рассматриваемой категории файлов на предмет выявления в них вредоносных признаков.

Для проведения анализа возможно использование статических и динамических методов.

Статический анализ незаменим при исследовании исходного или интерпретируемого кода, не защищенного механизмами обфускации и полиморфизма. В случае применения данных механизмов защиты статический анализ не эффективен.

Единственным вариантом в статическом анализе, гарантирующим обнаружение опасного кода в бинарном программном файле, является его полное дизассемблирование с последующим анализом. Однако дизассемблирование ввиду его сложности и трудоемкости нельзя рекомендовать для широкого использования.

В отличие от статического, динамический анализ позволяет в режиме реального времени отслеживать действия вредоносной программы и вносимые ею изменения в операционную среду и является более эффективным ввиду меньшей сложности и трудозатратности.

Данные подходы могут использоваться как по отдельности, так и вместе, дополняя друг друга при анализе.

Из всех возможных форматов документов мы сконцетрируемся на \textbf{doc}, а по итогам работы создадим и опишем модели, позволяющие систематизировать и в дальнейшем автоматизировать обработку характеристик, полученных только методом динамического анализа работающих процессов.

В первой главе мы опишем, какие объекты были включены в исследование.
Также мы опишем, каким образом документ, априори не предназначенный для вредоносных действий, может создавать реальную угрозу при работе с ним.

Во второй главе мы рассмотрим один из способов сбора динамической информации о запущенных процессах в операционной системе Microsoft Windows. 
По сформированной выборке из файлов формата \textbf{doc} мы будем собирать информацию с помощью утилиты PROCMON.
Также будет описан формат файла, полученного в результате и содержащего необходимые нам данные.

Третья глава посвящена описанию алгоритмов машинного обучения, которые нам понадобятся для дальнейшей работы над созданием метода классификации вредоносных программ.
В начале главы мы сформулируем главную проблему классификации и дадим необходимые определения.
Так как мы хотим классифицировать файлы произвольной структуры, далее, мы рассмотрим принципы, позволяющие работать с произвольно сложными объектами в качестве входных данных для алгоритмов машинного обучения.
Также будут описаны несколько моделей классификации, пользуясь которыми, мы в дальнейшем будем решать задачу распознавания вредоносных объектов.
В качестве таких алгоритмов, позволяющих создавать математические модели, мы рассмотрим метод K ближайших соседей и логистическую регрессию.
В завершении главы мы опишем способы оценки качества полученных моделей классификации и сравним несколько алгоритмов между собой на традиционном для статистики примере: задачи распознавания цветков ириса.

В четвёртой главе мы воспользуемся информацией, полученной в предыдущих главах, и создадим набор из двух алгоритмов, позволяющих классифицировать объекты из собранной выборки.

