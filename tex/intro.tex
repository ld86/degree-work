\likechapter{ВВЕДЕНИЕ}

Современный мир уже невозможно представить без взаимодействия с компьютером, а найти сферу жизни человека, до которой ещё не добрались информационные технологии, с каждым годом становится всё тяжелей.
Если раньше компьютеры связывались напрямую с потребностями в работе и в силу своей цены были доступны только большим компаниям, то сейчас они есть практически у каждого человека и зачастую не в одном экземпляре.
Также с развитием содержания меняется и форма.
Современные смартфоны по вычислительным способностям можно сопоставить с настольными компьютерами, которые использовались не более десятка лет назад, а такое направление как “носимая электроника” буквально стирает грань между техникой и человеком.
Компании, которые раньше сталкивались с проблемами масштабирования протокола IP, являющимся основным для обмена сообщениями по сети, всё чаще переходят на обновлённый протокол IPv6, позволяющий буквально присвоить интернет адрес каждой песчинке на Земле.

Но как и в любой другой сфере, в компьютерной индустрии существуют субъекты, использующие незаконные методы для извлечения прибыли.
Однако прибыль не всегда выражается в обладании некоторым физическим объектом.
Всё чаще гораздо большую ценность имеет информация, хранящаяся на компьютерных носителях, а растущая распространённость техники, контролируемой посредством некоторого программного кода, всё сильнее пугает потенциальными последствиями её захвата.

Если в реальном мире средством захвата часто выступает то или иное оружие, то в сфере информационных технологий для кражи или уничтожения информации используется вредоносное программное обеспечение.
Существует множество различных типов вредоносных программ, отличающихся различными характеристиками, например от поставленной перед вирусом задачи или способа исполнения программного кода.
Более того у каждой крупной антивирусной компании есть формальная классификация такого рода программ.

Практически исследование вредоносного кода разделяют на несколько этапов:
\begin{itemize}
\item статический анализ
\item динамический анализ
\end{itemize}
Статический анализ подразумевает получение информации, необходимой для определения типа вируса, без непосредственного запуска вредоносного кода.
Динамический же анализ напротив полностью основан на исследовании процесса работы запущенной программы.
У каждого из методов есть свои достоинства и недостатки.
Данные этапы могут использоваться как по отдельности так и вместе, дополняя друг друга при анализе.

Цель данной работы заключается в исследовании отдельного направления вирусной индустрии, включающего в себя объекты, распространяющиеся под видом документов, например это могут быть файлы формата \textbf{.doc}, \textbf{.pdf} или \textbf{.xls}, при этом речь пойдёт не о известных многим макро-вирусах, а о более сложных подвидах.
Из всех возможных форматов документов мы сконцетрируемся на \textbf{.doc}, а по итогам работы будут созданы и описаны модели, позволяющие систематизировать и в дальнейшем автоматизировать обработку характеристик, полученных только методом динамического анализа работающих процессов.

В первой главе мы опишем, какие объекты были включены в исследование.
Также мы опишем каким образом документ, априори не предназначенный для вредоносных действий, может создавать реальную угрозу при работе с ним.

Во второй главе мы рассмотрим один из способов сбора динамической информации о запущенных процессах в операционной системе Microsoft Windows. 
По сформированной выборке из файлов формата \textbf{.doc} мы будем собирать информацию с помощью утилиты PROCMON.
Также будет описан формат файла, полученного в результате и содержащего необходимые нам данные.

Третья глава посвящена описанию алгоритмов машинного обучения, которые нам понадобятся для дальнейшей работы над созданием метода классификации вредоносных программ.
В начале главы мы сформулируем главную проблему классификации и дадим необходимые определения.
Так как мы хотим классифицировать файлы произвольной структуры, далее, мы рассмотрим принципы, позволяющие работать с произвольно сложными объектами в качестве входных данных для алгоритмов машинного обучения.
Также будут описаны несколько моделей классификации, пользуясь которыми, мы в дальнейшем будем решать задачу распознавания вредоносных объектов.
В качестве таких алгоритмов, позволяющих создавать математические модели, мы рассмотрим метод K ближайших соседей и логистическую регрессию.
В завершении главы мы опишем способы оценки качества полученных моделей классификации и сравним несколько алгоритмов между собой на традиционном для статистики примере: задачи распознавания цветков ириса.

В четвёртой главе мы воспользуемся информацией, полученной в предыдущих главах, и создадим набор из двух алгоритмов, позволяющих классифицировать объекты из собранной выборки.

