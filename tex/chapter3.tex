\chapter{ПОДХОД МАШИННОГО ОБУЧЕНИЯ}

В этой главе будут даны основные определения, а также  сформулирована общая задача классификации.

Машинного обучение — это подраздел искусственного интелекта, находящийся на стыке статистики и компьютерных наук. Главная цель машинного обучения — создание методов, позволяющих строить математические модели и способных обучаться по прецедентам. Под обучением стоит понимать процесс аппроксимации заранее неизвестной функции по набору точек из её области определения и известных значений, полученных для каждой такой точки.

В нашем случае точки — это файлы формата .doc, а неизвестная функция отображает каждый такой файл во множество, состоящее из двух элементов {“вредоносный файл”, “безопасный файл”}.

Далее, мы более формально опишем, каким образом объекты  из реальной жизни можно рассматривать как точки, принадлежащие области определения неизвестной нам функции, и опишем подходы для извлечения зависимостей, c помощью которых в следующей главе будем решать задачи распознавания вредоносных программ.

\section{Обобщённая задача классификации}

Пусть $X$ — это множество объектов, а $Y$ — это множество классов.
Существует неизвестное нам отображение $F : X \to Y$, ставящее каждому объекту из $x \in X$ в соответствие метку класса из $y \in Y$.
Несмотря на то, что само отображение нам неизвестно, для некоторого набора элементов $\{ x_1, \dots , x_n \} \subset X$ мы можем получить значения $F$ в этих точках  $y_i = F(x_i)$ $i \subset \{ 1, \dots, n \}$.
