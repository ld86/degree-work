\chapter{ФОРМИРОВАНИЕ ВЫБОРКИ}

Как говорилось в введении, в качестве объектов исследования будут выступать файлы сложно формата \textbf{.doc}.
Данный формат используется для хранения информации о текстовых, графических документах.
Основным программным продуктом, позволяющим создавать и редактировать файлы формата \textbf{.doc}, является Microsoft Word.
Файлы данного формат имеют бинарную структуру, схожую по строению со структурой файловых систем. \cite{doc_format}

Основной задачей данной работы является описание и построение модели, способной самостоятельно и автоматически определять степень вредоности поступающих к нам извне новых файлов формата \textbf{.doc}.
Для достижения данной цели мы будем использовать комбинацию известных алгоритмов, позволяющих учитывать предыдущий опыт -- так называемых обучаемых алгоритмов.
В нашем случае под опытом стоит понимать наличие двух наборов файлов: множество безопасных для работы документов и множество вредоносных документов.
Объеденение этих двух наборов мы в дальнейшем будем называть выборкой объектов, подразумевая что файлы были выбраны из некоторого нам неизвестного множества всех возможных файлов формата \textbf{.doc}. 
Процесс получения данных наборов называется разметкой, почти всегда данный процесс происходит в ручном режиме.
Тем не менее обычно нас не интересуют каким способом было получено разделение файлов на разные типы, но при этом для нас очень важно, что вероятность ошибочного размещение документа в не подходящем классе была близка к нулю.
Иначе основываясь на плохо размеченной выборке объектов мы можем создать ошибочную модель, допускающую при работе заметное число ошибок.

В отличии от ручного анализа, когда все действия выполняемые программой при работе с документом анализируются человеком, нам для успешного создания работающей модели важно наличие файлов различных типов.
Таким образом, как первый шаг, мы опишем процесс создания выборки объектов, которые в дальнейшем будут использоваться в качестве исходных данных для алгоритмов обучения.

\section{Вредоносные файлы}

\section{Безопасные файлы}