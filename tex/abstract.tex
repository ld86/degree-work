\likechapterheading{РЕФЕРАТ}

Мельник Б. В. ДИНАМИЧЕСКИЙ АНАЛИЗ ПОТЕНЦИАЛЬНО ОПАСНЫХ ФАЙЛОВ СЛОЖНЫХ ФОРМАТОВ НА ПРЕДМЕТ ИХ ОТНОСИМОСТИ К ВРЕДОНОСНЫМ ПРОГРАММАМ, дипломная работа: стр. XX, рис. XX, табл. XX, библ. XX назв.

Ключевые слова: ВРЕДОНОСНОЕ ПРОГРАММНОЕ ОБЕСПЕЧЕНИЕ, PROCMON, MICROSOFT WORD, ДИНАМИЧЕСКИЙ АНАЛИЗ, МАШИННОЕ ОБУЧЕНИЕ, АЛГОРИТМ DYNAMIC TIME WARPING.

Объект исследования - файлы сложного формата \textbf{.doc}, используемые для работы в программном продукте Microsoft Word.
Цель работы - исследование процесса работы программного пакета Microsoft Word, описание работы вредоносных программ, распространяющихся через файлы сложных форматов, создание методов, позволяющих определять степень вредоносности файлов формата \textbf{.doc}.

В результате анализа работы была изучена работа вредоносных программ, распространяющихся через уязвимости файловых форматов. Также предложено несколько алгоритмов, использующих техники машинного обучения и позволяющих с приемлемым качеством классифицировать вредоносные объекты.