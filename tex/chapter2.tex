\chapter{СБОР ДАННЫХ}

В этом главе мы опишем метод и используемое программное обеспечение, с помощью которого был осуществлён сбор данных по выборке объектов формата \textbf{.doc}.

\section{Рабочее окружение}

При работе с объектами, которые потенциально могут приносить вред, важно построить безопасное для работы окружение.
Как было сказано ранее существует две стратегии исследоавния вредоносновного кода: статический анализ и динамический анализ.
Статический анализ подразумевает исследование объектов без фактического запуска вредоносного программного код, например к такому виду анализа относится сигнатурный анализ.
Таким образом статический анализ не требует введения сложных мер для достижения безопасности исследования, зачастую достаточно только обеспечить изолированое хранение исследуемых объектов.
Динамический анализ является более мощным инструментов, позволяющим получать гораздо больше информации о потенциально вредосноных действиях исследуемых объектов, но при этом подразумевает наличие этапа запуска вредосного программного кода.
Даже при принятии мер предосторожности, запуск вредоносных программ и последующий анализ работы на основной для исследователя системе может вылиться в непреднамеренное заражение рабочей системы и возникновению различных угроз для хранимой информации.

Один из возможных способов достижения безопасности при работе с опасными объектами является выделение отдельного компьютера.
Такой способ надёжен, но не очень практичен. Во-первых нужно при исследовании иметь в наличии ещё одно физическое устройство, что не всегда возможно, во-вторых работа вредосновных програм может вносить непоправимые изменения в работу служебных файлов и системных процессов, таким образом довольно часто придётся тратить время на восстановление окружения, в-третих при последовательном исследовании нескольких объектов результат работы первой программы может влиять на испольнение следующих, что нарушает независимость испытаний и вносит смещение в полученные выводы.

С увеличением мощности компьютерной техники и развитием апаратной виртуализации, современные вирусные аналитики всё чаще использует продукты компаний, выпускающих программное обеспечение для создания и работы с виртуальными машинами. 
Данная технология лишена всех недостатков, присущих отдельному компьютеру для исследования.
Для анализа вредоносных объектов в данной работе была использована виртуальная машина компании VMWare \footnote{VMware Player -- https://www.vmware.com/ru/products/player}, настроеная таким образом, что любые изменение сделанные во время работы машины фактически не вносились в рабочий образ системы.
Это позволяет перед началом анализа очередного объекта иметь идентичное рабочее окружение.
В качестве операционной системые использовалась Microsoft Windows XP Professional версии 2002 с установленным Service Pack 3.
Для сбора динамических характеристик 

\section{Процесс сбора данных}

