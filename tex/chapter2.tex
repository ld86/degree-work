\chapter{СБОР ДАННЫХ}

В этом главе мы опишем метод и используемое программное обеспечение, с помощью которого был осуществлён сбор данных по выборке объектов формата \textbf{.doc}.

\section{Рабочее окружение}

При работе с объектами, которые потенциально могут приносить вред, важно построить безопасное для работы окружение.
Как было сказано ранее существует две стратегии исследоавния вредоносновного кода: статический анализ и динамический анализ.
Статический анализ подразумевает исследование объектов без фактического запуска вредоносного программного код, например к такому виду анализа относится сигнатурный анализ.
Таким образом статический анализ не требует введения сложных мер для достижения безопасности исследования, зачастую достаточно только обеспечить изолированое хранение исследуемых объектов.
Динамический анализ является более мощным инструментов, позволяющим получать гораздо больше информации о потенциально вредосноных действиях исследуемых объектов, но при этом подразумевает наличие этапа запуска вредосного программного кода.
Даже при принятии мер предосторожности, запуск вредоносных программ и последующий анализ работы на основной для исследователя системе может вылиться в непреднамеренное заражение рабочей системы и возникновению различных угроз для хранимой информации.

Один из возможных способов достижения безопасности при работе с опасными объектами является выделение отдельного компьютера.
Такой способ надёжен, но не очень практичен. Во-первых нужно при исследовании иметь в наличии ещё одно физическое устройство, что не всегда возможно, во-вторых работа вредосновных програм может вносить непоправимые изменения в работу служебных файлов и системных процессов, таким образом довольно часто придётся тратить время на восстановление окружения, в-третих при последовательном исследовании нескольких объектов результат работы первой программы может влиять на испольнение следующих, что нарушает независимость испытаний и вносит смещение в полученные выводы.

С увеличением мощности компьютерной техники и развитием апаратной виртуализации, современные вирусные аналитики всё чаще использует продукты компаний, выпускающих программное обеспечение для создания и работы с виртуальными машинами. 
Данная технология лишена всех недостатков, присущих отдельному компьютеру для исследования.
Для анализа вредоносных объектов в данной работе была использована виртуальная машина компании VMWare \footnote{VMware Player -- https://www.vmware.com/ru/products/player}, настроеная таким образом, что любые изменение сделанные во время работы машины фактически не вносились в рабочий образ системы.
Это позволяет перед началом анализа очередного объекта иметь идентичное рабочее окружение.

\begin{figure}[ht]
	\centering
    \begin{subfigure}[b]{1\textwidth}
    \centering
        \includegraphics[scale=0.5]{vmware.jpg}        
    \end{subfigure}
 
    \caption{Главное окно программы для работы с виртуальными машинами VMWare Player}
    \label{fig_parsetree}
\end{figure}

В качестве операционной системые использовалась Microsoft Windows XP Professional версии 2002 с установленным Service Pack 3.
Для сбора динамических характеристик собранные файлы открывались с помощью Microsoft Office Word 2003 версии 11.5604.5606.

Далее будут описаны инструменты для сбора динамических характеристик запущенных программ и формат получившихся данных.

\section{Процесс сбора данных}

Основным инструментом для получение промежуточных данных об объектах в данном исследовании является утилита PROCMON \footnote{Process Monitor -- https://technet.microsoft.com/en-us/library/bb896645.aspx}.

\begin{figure}[ht]
	\centering
    \begin{subfigure}[b]{1\textwidth}
    \centering
        \includegraphics[scale=0.5]{procmon_main_window.png}        
    \end{subfigure}
 
    \caption{Главное окно утилиты PROCMON}
    \label{fig_parsetree}
\end{figure}

Данная утилита позволяет собирать большой спектр различных событий, возникающих в процессе работы процессов, такие как:
\begin{itemize}
\item вызовых функций WinAPI
\item чтение или изменение системного реестра
\item работа с дисковой подсистемой
\end{itemize}

По умолчанию PROCMON записывает все события, просходящие в системе. 
Нам же необходимо получать действия, совершаемые только Microsoft Word.
Спецально для этого предусмотрена функциональность фильтрации собранных событий по различным признакам.
При старте PROCMON предлагает выбрать желаемые значения этих фильтров.

\begin{figure}[ht]
	\centering
    \begin{subfigure}[b]{1\textwidth}
    \centering
        \includegraphics[scale=0.5]{procmon_filters.png}        
    \end{subfigure}
 
    \caption{Окно установки фильтров утилиты PROCMON}
    \label{fig_parsetree}
\end{figure}

Нам нужны события, создаваемые программой с именем \textbf{WINWORD.exe} и как можно видеть на рисунке 2.2 мы добавили соответствующий фильтр на название процесса.
Также нам нужно определиться с составом данных, которые будут сохраняться для дальнейшего использования.
PROCMON позволяет выбрать из большого числа возможных полей.

\begin{figure}[ht]
	\centering
    \begin{subfigure}[b]{1\textwidth}
    \centering
        \includegraphics[scale=0.5]{procmon_columns.png}        
    \end{subfigure}
 
    \caption{Данные о процессах, предоставляемые утилитой PROCMON}
    \label{fig_parsetree}
\end{figure}

На рисунке 2.3 отмечены необходимые для нас поля:
\begin{itemize}
\item Operation -- операция, совершаемая объектов
\item Relative time -- время совершения операции с момента запуска программы
\item Duration -- время, которое потребовалось для совершения операции
\item Process ID -- уникальный идентификатор процесса
\end{itemize}

После того как были установлены правильные фильтры и выбраны нужные поля, производится открытие каждого файла из выборки.
При этом в главном окне утилиты PROCMON можно наблюдать захваченные события.

\begin{figure}[ht]
	\centering
    \begin{subfigure}[b]{1\textwidth}
    \centering
        \includegraphics[scale=0.5]{procmon_events.png}        
    \end{subfigure}
 
    \caption{Процесс сбора динамических событий}
    \label{fig_parsetree}
\end{figure}

После достижения достаточного количества собранной информации, она сохраняется в текстовые файлики для дальнешего анализа.
Формат этих файликов очень просто, он представляет собой значения, разделённые заранее обговорённым разделителем, например запятой \footnote{CSV -- Comma-separated values}.

\begin{figure}[ht]
	\centering
    \begin{subfigure}[b]{1\textwidth}
    \centering
        \includegraphics[scale=0.5]{csv.png}        
    \end{subfigure}
 
    \caption{Несколько строк из итогового файла с записанными события}
    \label{fig_parsetree}
\end{figure}

В дальнейшем  для каждого объекта по собранным с помощью данной методики файлов будут выделяться формальные признаки и использоваться для построения итоговой модели.